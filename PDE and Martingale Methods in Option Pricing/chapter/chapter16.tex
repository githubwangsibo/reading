\chapter{Elements of Malliavin calculus}

\section{Stochastic derivative}

\begin{proposition}
$\forall n \in \mathbb{N}$,  $W_{T} \in \mathcal{S}_{n}$. 
\end{proposition}

\begin{proof}
Given $n$, if $\varphi(\Delta_{n}) = \varphi(x_{1}, \dots, x_{2^{n}}) \coloneqq x_{1} + \dots + x_{2^{n}}$, then $\varphi(\Delta_{n}) \in \mathcal{S}_{n}$. Therefore $W_{T} \in \mathcal{S}_{n}$. 
\end{proof}

\begin{proposition}
$\forall n \in \mathbb{N}$,  $\mathcal{S}_{n} \subseteq \mathcal{S}_{n+1}$. 
\end{proposition}

\begin{proof}
The partition length is $\frac{1}{2^{n}}$ for each $n \in \mathbb{N}$. 
\end{proof}

\begin{proposition}
$\mathcal{S}$ is dense in $L^{p}(\Omega, \mathcal{F}^{W}_{T})$. 
\end{proposition}

\begin{proof}
To-do. 
\end{proof}

\begin{note}
Given a filtered measure space $(\Omega, \mathcal{A}, \mathcal{A}_{n}, \mu)$, for each $X \in \mathcal{S}_{n}$,  $X$ and $D_{t}X$ are both measurable functions: $\Omega \to \mathbb{R}$. 
\end{note}












